\documentclass{article}
\usepackage[margin = 1in]{geometry}
% packages %
\usepackage{graphicx}
\usepackage{amsmath}
\usepackage{amssymb}
\usepackage[dvipsnames]{xcolor} 
\usepackage{caption}
\usepackage{hologo}
\usepackage{fontspec}
\usepackage{unicode-math}
\usepackage[hidelinks]{hyperref}
\usepackage[type={CC},modifier={by-nc-nd},version={4.0}]{doclicense}
\usepackage{zennote}
% format %
\setlength{\parindent}{0pt}
\linespread{1.3}
\renewcommand{\contentsname}{\centering Table of Contents}
% font %
\setmainfont{TeX Gyre Pagella}
\setmathfont{TeX Gyre Pagella Math}
\newfontfamily\textcm{CMU Serif}
\hypersetup{
    colorlinks = True,
    allcolors = cyan
    }


% document %
\begin{document}

\titlebox{
    top-left = {{\bf{\textcm \LaTeX}}},
    top-right = {{\bf Packages}},
    bottom-left = {{\sl Author: Yifan Liang}},
    bottom-right = {{\sl Version: 1.0.0}},
    notenumber = {1},
    topic = {{\tt zennote} for Fancy Notes},
    type = {Manual}
    }

{{\bf Note: }This documentation needs to be compiled in {\textcm \hologo{XeLaTeX}} or {\textcm \hologo{LuaLaTeX}}.}

\thispagestyle{empty}
\tableofcontents

\newpage
\setcounter{page}{1}

\section{Introduction}
\subsection{Basic Functions}
This {\tt zennote} package aims to provide you with an easy interface to speed up the process when organizing and producing elegant notes.\\
\\
All the tables, figures, equations, and listings are labeled according to the {\tt notenumber} with the \verb+\titlebox+ command.\\
\\
The {\tt noteframe} environment helps you generate fancy colored boxes to emphasize the important information (e.g. Theorems, Equations, Proof, etc.) in your document. You can customize the style and color to denote different categories, too.

\subsection{License}
This work is licensed under \href{https://creativecommons.org/licenses/by-nc-nd/4.0/}{Creative Commons "Attribution-NonCommercial-NoDerivatives 4.0 International" license}.\\
\\
This license requires that reusers give credit to the creator. It allows reusers to copy and distribute the material in any medium or format in unadapted form and for noncommercial purposes only.\\
\\
\begin{flushright}
    \doclicenseImage
\end{flushright}

\newpage

\section{Note Title}
{\tt zennote} provides \verb+\titlebox{<settings>}+ command to generate the following title box for your note, where {\tt settings} help you set the key information of your note, including:
\begin{itemize}
    \item \verb+top-left+, Text on Top-Left Corner;
    \item \verb+top-right+, Text on Top-Right Corner;
    \item \verb+bottom-left+, Text on Bottom-Left Corner;
    \item \verb+bottom-right+, Text on Bottom-Right Corner;
    \item \verb+notenumber+, Note Number;
    \item \verb+topic+, Note Topic;
    \item \verb+type+, Note Type (e.g. Chapter, Unit, Note).
\end{itemize}

\subsection{Usage}
\begin{flushleft}
\begin{verbatim}
\titlebox{
    top-left = {topleft},
    top-right = {topright},
    bottom-left = {bottomleft},
    bottom-right = {bottomright},
    notenumber = {1},
    topic = {topic},
    type = {Note}
}    
\end{verbatim}
\end{flushleft}

\subsection{Result}
\titlebox{
    top-left = {topleft},
    top-right = {topright},
    bottom-left = {bottomleft},
    bottom-right = {bottomright},
    notenumber = {1},
    topic = {topic},
    type = {Note}
} 

\newpage

\section{Note Highlighter}
{\tt zennote} provides \verb+noteframe+ environment to generate the following "highlighter boxes." Users can set the key values in options to customize the styles of the frame, which includes:
\begin{itemize}
    \item \verb+color+, The color of the frame, which can be chosen from  \verb+\usepackage[dvipsnames]{xcolor}+. The default color is \textcolor{cyan}{\bf cyan};
    \item \verb+style+, The edge of the frame, which can be chosen from  \verb+zero+, \verb+one+, and \verb+two+. The default edge style is \verb+one+;
    \item \verb+title+, The title of the frame.
\end{itemize}

\subsection{Examples}
\subsubsection{Usage 1}
\begin{flushleft}
\begin{verbatim}
\begin{noteframe}[title = {Default}]
    Cyan Box Example Content.
\end{noteframe}
\end{verbatim}
\end{flushleft}

\subsubsection{Result 1}
\begin{noteframe}[title = {Default}]
    Cyan Box Example Content.
\end{noteframe}

\begin{flushright}
{\sl (To be continued on the next page...)}
\end{flushright}

\newpage

\subsubsection{Usage 2}
\begin{flushleft}
\begin{verbatim}
\begin{noteframe}[
    title = {style: two and color: LimeGreen},
    color = {LimeGreen},
    style = {two}
]
    LimeGreen Box Example Content.
    \begin{align}
        a^2 + b^2 = c^2
    \end{align}
\end{noteframe}
\end{verbatim}
\end{flushleft}

\subsubsection{Result 2}
\begin{noteframe}[
    title = {style: two and color: LimeGreen},
    color = {LimeGreen},
    style = {two}
]
    LimeGreen Box Example Content.
    \begin{align}
        a^2 + b^2 = c^2
    \end{align}
\end{noteframe}

\begin{flushright}
{\sl (To be continued on the next page...)}
\end{flushright}

\newpage

\subsubsection{Usage 3}
\begin{flushleft}
\begin{verbatim}
\begin{noteframe}[style = {zero}]
    Cyan Box Example Content.
    \begin{center}
        \includegraphics[width = 5cm]{example-image-a}
        \captionof{figure}{this is a picture.}
    \end{center}
\end{noteframe}
\end{verbatim}
\end{flushleft}

\subsubsection{Result 3}
\begin{noteframe}[style = {zero}]
    Cyan Box Example Content.
    \begin{center}
        \includegraphics[width = 5cm]{example-image-a}
        \captionof{figure}{this is a picture.}
    \end{center}
\end{noteframe}

\subsection{Caveats}
It is crucial to write \verb+\usepackage{zennote}+ \textbf{\textit{after}} \verb+\usepackage[dvipsnames]{xcolor}+ to prevent potential error when coloring the frame.

\end{document}